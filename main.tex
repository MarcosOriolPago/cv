\documentclass[11pt,a4paper]{article}

% ----------- Packages -----------
\usepackage[a4paper,margin=1.5cm]{geometry}
\usepackage{fontawesome5}
\usepackage{graphicx}
\usepackage{tikz}
\usepackage{pgfplots}
\usepackage{helvet}
\usepackage{hyperref}
\usepackage{xcolor}
\usepackage{enumitem}
\usepackage{wrapfig}

% ----------- Settings -----------
\renewcommand{\familydefault}{\sfdefault}
\pagestyle{empty}

% Colors
\definecolor{accent}{HTML}{1F77B4} % change this to your accent color
\definecolor{textgray}{HTML}{444444}

\hypersetup{
  colorlinks=true,
  urlcolor=accent,
  linkcolor=accent
}

% ----------- Header -----------
\newcommand{\cvheader}[5]{%
  \begin{tikzpicture}[remember picture,overlay]
    \fill[accent!10] (current page.north west) rectangle ([yshift=-4cm]current page.north east);
    \node[anchor=north west, xshift=1.5cm, yshift=-1cm] (photo) {\includegraphics[width=2.8cm,height=2.8cm,keepaspectratio,clip,trim=0 0 0 0,frame]{#1}};
    \node[anchor=north west, align=left, xshift=5cm, yshift=-1cm, text=textgray] (info) {%
      {\Huge \textbf{#2}}\\[4pt]
      {\large #3}\\[6pt]
      \faEnvelope\ \href{mailto:#4}{#4}\quad
      \faLinkedin\ \href{#5}{#5}
    };
  \end{tikzpicture}
  \vspace{2.5cm}
}

% ----------- Section -----------
\newcommand{\cvsection}[1]{%
  \vspace{0.6cm}
  {\Large\color{accent}\textbf{#1}}\par
  \vspace{0.3cm}\hrule height 1pt color{accent!50}\vspace{0.4cm}
}

% ----------- Skill bar -----------
\newcommand{\skillbar}[2]{%
  \noindent
  \begin{tikzpicture}
    \fill[gray!20] (0,0) rectangle (10,0.25);
    \fill[accent!80] (0,0) rectangle (#2*10/100,0.25);
    \node[anchor=west,text=textgray] at (10.5,0.12) {\small #1};
  \end{tikzpicture}\vspace{2pt}
}

% ----------- Radar chart -----------
\pgfplotsset{compat=1.18}
\newcommand{\skillsradar}{
\begin{tikzpicture}[scale=0.8]
\begin{polaraxis}[
  hide axis,
  xtick={0,72,144,216,288},
  xticklabels={Programming,Design,Leadership,Communication,Problem Solving},
  ymin=0,ymax=5,
  grid=none,
  major grid style={thin,accent!50},
  minor grid style={dotted,accent!30},
]
\addplot[accent!70,fill=accent!40,opacity=0.7,mark=*] coordinates {
  (0,4.8) (72,3.5) (144,3.8) (216,4.0) (288,4.5) (360,4.8)
};
\end{polaraxis}
\end{tikzpicture}
}

% ----------- Timeline -----------
\newcommand{\timelineitem}[4]{%
  % #1 year, #2 title, #3 company, #4 description
  \node[anchor=west,text width=6cm,align=left] at (1, -#1*2.5) {
    \textbf{#2} \\[-2pt]
    {\small\textit{#3}} \\[-2pt]
    {\footnotesize #4}
  };
  \draw[accent, thick] (0, -#1*2.5) circle (4pt);
  \draw[accent!50, thick] (0, -#1*2.5) -- (0, -(#1+1)*2.5);
}

% ===============================
%             DOCUMENT
% ===============================
\begin{document}
\cvheader{photo.jpg}{John Doe}{Software Engineer}{john@example.com}{linkedin.com/in/johndoe}

% --- Left side main content ---
\cvsection{Profile}
I am a passionate software engineer with a focus on front-end technologies, UI design, and data visualization. 
I love building interfaces that communicate information beautifully and efficiently.

\cvsection{Skills}
\skillsradar

\vspace{0.5cm}
\skillbar{React / TypeScript}{90}
\skillbar{Node.js / Express}{85}
\skillbar{Python / Data Visualization}{80}
\skillbar{LaTeX / Documentation}{70}

\cvsection{Languages}
\skillbar{English}{95}
\skillbar{Spanish}{85}
\skillbar{German}{60}

% --- Timeline ---
\cvsection{Experience}
\begin{tikzpicture}[x=1cm,y=1cm]
  % Vertical line
  \draw[accent!40, ultra thick] (0,0) -- (0,-10);
  % Items
  \timelineitem{0}{Senior Frontend Developer}{TechCorp Inc.}{React, TypeScript, and design systems for analytics dashboards.}
  \timelineitem{1}{UI Engineer}{Designify Studio}{Built reusable component libraries and data-driven visualizations.}
  \timelineitem{2}{Software Engineer}{OpenDev Labs}{Developed visualization tools for open data platforms.}
  \timelineitem{3}{Intern Developer}{InnovateX}{Worked on web accessibility and responsive layouts.}
\end{tikzpicture}

\cvsection{Education}
\textbf{B.Sc. in Computer Science}, University of Example (2017–2021)

\end{document}
